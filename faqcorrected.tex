%%%%%%%%%%%%%%%%%%%%%%%%%%%%%%%%%%%%%%%%%
% Frequently Asked Questions
% LaTeX Template
% Version 1.0 (22/7/13)
%
% This template has been downloaded from:
% http://www.LaTeXTemplates.com
%
% Original author:
% Adam Glesser (adamglesser@gmail.com)
%
% License:
% CC BY-NC-SA 3.0 (http://creativecommons.org/licenses/by-nc-sa/3.0/)
%
%%%%%%%%%%%%%%%%%%%%%%%%%%%%%%%%%%%%%%%%%

\documentclass[11pt]{article}

\usepackage[margin=1in]{geometry} % Required to make the margins smaller to fit more content on each page
\usepackage[linkcolor=blue]{hyperref} % Required to create hyperlinks to questions from elsewhere in the document
\hypersetup{pdfborder={0 0 0}, colorlinks=true, urlcolor=blue} % Specify a color for hyperlinks
\usepackage{todonotes} % Required for the boxes that questions appear in
\usepackage{tocloft} % Required to give customize the table of contents to display questions
\usepackage{microtype} % Slightly tweak font spacing for aesthetics
\usepackage{palatino} % Use the Palatino font



\usepackage[utf8]{inputenc} 
\usepackage[french]{babel} 



\usepackage{listings} % Required for inserting code snippets


\lstdefinestyle{customc}{
  belowcaptionskip=1\baselineskip,
  breaklines=true,
  frame=L,
  xleftmargin=\parindent,
  language=C,
  showstringspaces=false,
  basicstyle=\footnotesize\ttfamily,
  keywordstyle=\bfseries\color{green!40!black},
  commentstyle=\itshape\color{purple!40!black},
  identifierstyle=\color{blue},
  stringstyle=\color{orange},
}

\lstdefinestyle{customasm}{
  belowcaptionskip=1\baselineskip,
  frame=L,
  xleftmargin=\parindent,
  language=[x86masm]Assembler,
  basicstyle=\footnotesize\ttfamily,
  commentstyle=\itshape\color{purple!40!black},
}

\lstset{escapechar=@,style=customc}


\setlength\parindent{0pt} % Removes all indentation from paragraphs

% Create and define the list of questions
\newlistof{questions}{faq}{\large List of Frequently Asked Questions} % This creates a new table of contents-like environment that will output a file with extension .faq
\setlength\cftbeforefaqtitleskip{4em} % Adjusts the vertical space between the title and subtitle
\setlength\cftafterfaqtitleskip{1em} % Adjusts the vertical space between the subtitle and the first question
\setlength\cftparskip{.3em} % Adjusts the vertical space between questions in the list of questions

% Create the command used for questions
\newcommand{\question}[1] % This is what you will use to create a new question
{
\refstepcounter{questions} % Increases the questions counter, this can be referenced anywhere with \thequestions
\par\noindent % Creates a new unindented paragraph
\phantomsection % Needed for hyperref compatibility with the \addcontensline command
\addcontentsline{faq}{questions}{#1} % Adds the question to the list of questions
\todo[inline, color=blue!40]{\textbf{#1}} % Uses the todonotes package to create a fancy box to put the question
\vspace{1em} % White space after the question before the start of the answer
}

% Uncomment the line below to get rid of the trailing dots in the table of contents
%\renewcommand{\cftdot}{}

% Uncomment the two lines below to get rid of the numbers in the table of contents
%\let\Contentsline\contentsline
%\renewcommand\contentsline[3]{\Contentsline{#1}{#2}{}}



\begin{document}

%----------------------------------------------------------------------------------------
%	TITLE AND LIST OF QUESTIONS
%----------------------------------------------------------------------------------------

\begin{center}
\Huge{\bf \emph{NoSQL 2015 \\ TD 2}} % Main title
\end{center}

\begin{center}
Professeurs 

\bigskip 


Houda Chabbi Drissi \\et \\Benoît Perroud \\
\end{center}

\begin{center}
\begin{figure}[h!]

  \centering
    \includegraphics[width=0.4\textwidth]{Images/Logo_HEIA-FR_site}
     
\end{figure}
\end{center}



\begin{figure}[h!]

  \centering
    \includegraphics[width=1\textwidth]{Images/Hadoop-tutorial.png}
     
\end{figure}

\bigskip

\begin{center}
 


\begin{figure}[h!]

  \centering
    \includegraphics[width=0.5\textwidth]{Images/DAPLab-Title.png}
     
\end{figure}
\end{center}


\begin{center} 
Assistant : Christophe Bovigny
\end{center}

\listofquestions % This prints the subtitle and a list of all of your questions

\newpage % Comment this if you would like your questions and answers to start immediately after table of questions

%----------------------------------------------------------------------------------------
%	QUESTIONS AND ANSWERS
%----------------------------------------------------------------------------------------

% \question{Prérequis}\label{new-question}
% 
% 
% Tous les étudiants doivent envoyer une clé publique ssh à l'addresse email : bovignyc@gmail.com afin de pouvoir se connecter et s'exercer sur le cluster DAPLAB
% \\
% Génération de la clé publique:
% \\
% \lstinputlisting[language=sh]{Scripts/sshkey.sh}
% 
% Clé à envoyer :
% \\
% \lstinputlisting[language=sh]{Scripts/seekey.sh}
% 
% Connection au Cluster :
% 
% Votre login est la première lettre de votre prenom suivi par votre nom : (Exemple : Christophe Bovigny = cbovigny). Durant ce travail dirigé tous les exemples seront produits 
% avec l'utilisateur cbovigny. 
% %\lstinputlisting[language=Python]{Scripts/login.sh}
% 
% \lstinputlisting[language=sh,mathescape]{Scripts/login.sh}
%  % The first argument is the script location/filename and the second is a caption for the listing
% %\begin{verbatim}
% %\question{A question that needs answering}\label{question-label}
% 
% %The answer to this question.
% %\end{verbatim}

%------------------------------------------------

%\question{Why is there a label in the code for \hyperref[new-question]{the previous question}?}\label{labels}


\question{Map Reduce}\label{new-question}

\begin{center}
{\bf \emph{Problème}} % Main title
\end{center}



Le but de cette partie est de comprendre la résolution d'un problème simple au moyen de l'algorithme map/reduce. Ici nous avons un fichier contenant des livres et nous voudrions
connaître combien de livres sont publiés par année.

Avant de se lancer dans le code, il est important de s'intéresser au fichier que l'on va utiliser : Copiez-le en locale dans votre home folder et ensuite analysez-le. (HDFS :/shared/tp2/BX-Books.csv). 
\paragraph{}

Quelles problèmes pourraient-être
rencontrer lors de l'insertion de ces données dans une table ?

\paragraph{}



Une classe java BookXMapper est définie comme ceci :

\lstinputlisting[language=java]{Scripts/BookXMapper.java}
\#\#\#\#\#\#\#\#\#\#\#\#\#\#\#\#\#\#\#\#\#\#\#\#\#\#\#\#\#\#\#\#\#\#\#\#\#\#\#\#\#\#\#\#\#\#\#\#\#\#\#\#\#\#\#\#\#\#\#\#\#\#\#\#\#\#\#\#\#\#\#\#\#\#\#\#\#\#\#\#\#\#\#\#\#\#\#\#\#\#\#\#\#\#

Solution :
%%%%%%%%%%%%%%%%%%%%%%%%% SOLUTION 1%%%%%%%%%%%%%%%%%%%%%%%%%%%%%%%%

Chaque ligne du fichier est découpée à chaque délimiteur ``;'' afin d'avoir un tableau de type ``String``. Le quatrième élément du tableau est le champ "Year-of-Publication'', lequel est la clé
du mapper.

%%%%%%%%%%%%%%%%%%%%%%%%%%%%%%%%%%%%%%%%%%%%%%%%%%%%%%%%%%%%%%%%%%%%

\#\#\#\#\#\#\#\#\#\#\#\#\#\#\#\#\#\#\#\#\#\#\#\#\#\#\#\#\#\#\#\#\#\#\#\#\#\#\#\#\#\#\#\#\#\#\#\#\#\#\#\#\#\#\#\#\#\#\#\#\#\#\#\#\#\#\#\#\#\#\#\#\#\#\#\#\#\#\#\#\#\#\#\#\#\#\#\#\#\#\#\#\#\#


Ensuite une méthode reduce est crée: expliquez-la:


\lstinputlisting[language=java]{Scripts/BookXReducer.java}
\#\#\#\#\#\#\#\#\#\#\#\#\#\#\#\#\#\#\#\#\#\#\#\#\#\#\#\#\#\#\#\#\#\#\#\#\#\#\#\#\#\#\#\#\#\#\#\#\#\#\#\#\#\#\#\#\#\#\#\#\#\#\#\#\#\#\#\#\#\#\#\#\#\#\#\#\#\#\#\#\#\#\#\#\#\#\#\#\#\#\#\#\#\#
\\Solution :
%%%%%%%%%%%%%%%%%%%%%%%%% SOLUTION 2%%%%%%%%%%%%%%%%%%%%%%%%%%%%%%%%

La méthode reduce prend en paramètres : key et Iterator$<$IntWritable$>$ values (valeurs groupées pour chaque key). Pour notre programme, nous utilisons à nouveau la clé de mapper comme l'output
du reducer et ajoutons chaque valeur de la liste. (Output de mapper est un new IntWritable(1)). Nous ajoutons toutes les occurences de new IntWritable(1) afin d'avoir le compte de livres
publiés pour chaque clé (année de publication)

%%%%%%%%%%%%%%%%%%%%%%%%%%%%%%%%%%%%%%%%%%%%%%%%%%%%%%%%%%%%%%%%%%%%
\#\#\#\#\#\#\#\#\#\#\#\#\#\#\#\#\#\#\#\#\#\#\#\#\#\#\#\#\#\#\#\#\#\#\#\#\#\#\#\#\#\#\#\#\#\#\#\#\#\#\#\#\#\#\#\#\#\#\#\#\#\#\#\#\#\#\#\#\#\#\#\#\#\#\#\#\#\#\#\#\#\#\#\#\#\#\#\#\#\#\#\#\#\#



Finalement une dernière classe BookXDriver est ajoutée et servira comme main program afin de lancer le job MapReduce.
\lstinputlisting[language=java]{Scripts/BookXDriver.java}


Maintenant nous allons lancer ce programme sur le cluster DAPLAB :


\lstinputlisting[language=sh,mathescape]{Scripts/launchmapreducesecond.sh}

Regardez l'output de votre map/reduce et expliquez-le


\question{HIVE}\label{new-question}

Maintenant nous allons résoudre le même problème que précédement, mais cette fois en utilisant la technologie HIVE.

Notre fichier BX-Books.csv n'est pas clean pour l'utiliser dans hive, c'est pourquoi la commande sed s'impose :


\lstinputlisting[language=sh]{Scripts/sed.sh}

Cette commande élimine les délimiteurs "`;`(semicolon) et les remplace par des ``\$\$\$'', et le motif "\&amp;" est remplacé par "AND". Cette commande élimine aussi le header du fichier.
(si on ne l'élimine pas Hive le considère comme data). Ce fichier corrigé se trouve dans /shared/tp2/BX-BooksCorrected.txt du système de fichier HDFS.



Maintenant nous allons lancer hive et créer une table BXDataSet.

A l'intérieur veuillez mettre ISBN (STRING), BookTitle (STRING), BookAuthor (STRING), YearOfPublication (STRING), Publisher (STRING), ImageURLS (STRING), ImageURLM (STRING) et ImageURLL (STRING) comme champ

\lstinputlisting[language=sh,mathescape]{Scripts/hivenotcorrected.sh}
\#\#\#\#\#\#\#\#\#\#\#\#\#\#\#\#\#\#\#\#\#\#\#\#\#\#\#\#\#\#\#\#\#\#\#\#\#\#\#\#\#\#\#\#\#\#\#\#\#\#\#\#\#\#\#\#\#\#\#\#\#\#\#\#\#\#\#\#\#\#\#\#\#\#\#\#\#\#\#\#\#\#\#\#\#\#\#\#\#\#\#\#\#\#
\\Solution :
%%%%%%%%%%%%%%%%%%%%%%%%% SOLUTION 3%%%%%%%%%%%%%%%%%%%%%%%%%%%%%%%%
\lstinputlisting[language=sh,mathescape]{Scripts/hive.sh}
%%%%%%%%%%%%%%%%%%%%%%%%%%%%%%%%%%%%%%%%%%%%%%%%%%%%%%%%%%%%%%%%%%%%
\#\#\#\#\#\#\#\#\#\#\#\#\#\#\#\#\#\#\#\#\#\#\#\#\#\#\#\#\#\#\#\#\#\#\#\#\#\#\#\#\#\#\#\#\#\#\#\#\#\#\#\#\#\#\#\#\#\#\#\#\#\#\#\#\#\#\#\#\#\#\#\#\#\#\#\#\#\#\#\#\#\#\#\#\#\#\#\#\#\#\#\#\#\#



Maintenant nous avons deux solutions pour créer la table, la première intéractivement et la seconde en une ligne de commande en mettant les instructions dans un fichier createtablehive.sql 
\lstinputlisting[language=sh,mathescape]{Scripts/hivelaunch.sh}

Ensuite il faut remplir la table bxdataset en utilisant le fichier BX-BooksCorrected\_CHRISTOPHE.txt se situant dans /shared/tp2/. Tout d'abord copiez ce fichier dans votre home hdfs et ensuite lancez
la commande ci-dessous.
\lstinputlisting[language=sh,mathescape]{Scripts/hiveloaddata.sh}


Afin d'obtenir le même résultat que pour le MapReduce, veuillez complèter la requête dans hive.

\lstinputlisting[language=sh,mathescape]{Scripts/hivesql.sh}
\#\#\#\#\#\#\#\#\#\#\#\#\#\#\#\#\#\#\#\#\#\#\#\#\#\#\#\#\#\#\#\#\#\#\#\#\#\#\#\#\#\#\#\#\#\#\#\#\#\#\#\#\#\#\#\#\#\#\#\#\#\#\#\#\#\#\#\#\#\#\#\#\#\#\#\#\#\#\#\#\#\#\#\#\#\#\#\#\#\#\#\#\#\#
\\Solution :
%%%%%%%%%%%%%%%%%%%%%%%%% SOLUTION 4%%%%%%%%%%%%%%%%%%%%%%%%%%%%%%%%
\lstinputlisting[language=sh,mathescape]{Scripts/hivesqlsol.sh}
%%%%%%%%%%%%%%%%%%%%%%%%%%%%%%%%%%%%%%%%%%%%%%%%%%%%%%%%%%%%%%%%%%%%
\#\#\#\#\#\#\#\#\#\#\#\#\#\#\#\#\#\#\#\#\#\#\#\#\#\#\#\#\#\#\#\#\#\#\#\#\#\#\#\#\#\#\#\#\#\#\#\#\#\#\#\#\#\#\#\#\#\#\#\#\#\#\#\#\#\#\#\#\#\#\#\#\#\#\#\#\#\#\#\#\#\#\#\#\#\#\#\#\#\#\#\#\#\#

Voilà votre requête dans hive donne exactement la même solution que votre MapReduce en java. 25 lignes se sont transformées en 3 lignes :)


\question{PIG}\label{new-question}


Maintenant nous allons effectuer la même chose mais en utilisant pig.

Nous allons lancer pig sur daplab, loader les données dans une variable BookXRecords, ensuite les grouper par année de publication et finalement concaténer le nombre de publications par année.
\lstinputlisting[language=sh,mathescape]{Scripts/pignotcorrected.sh}


\#\#\#\#\#\#\#\#\#\#\#\#\#\#\#\#\#\#\#\#\#\#\#\#\#\#\#\#\#\#\#\#\#\#\#\#\#\#\#\#\#\#\#\#\#\#\#\#\#\#\#\#\#\#\#\#\#\#\#\#\#\#\#\#\#\#\#\#\#\#\#\#\#\#\#\#\#\#\#\#\#\#\#\#\#\#\#\#\#\#\#\#\#\#
\\Solution :
%%%%%%%%%%%%%%%%%%%%%%%%% SOLUTION 5%%%%%%%%%%%%%%%%%%%%%%%%%%%%%%%%
\lstinputlisting[language=sh,mathescape]{Scripts/pig.sh}
%%%%%%%%%%%%%%%%%%%%%%%%%%%%%%%%%%%%%%%%%%%%%%%%%%%%%%%%%%%%%%%%%%%%
\#\#\#\#\#\#\#\#\#\#\#\#\#\#\#\#\#\#\#\#\#\#\#\#\#\#\#\#\#\#\#\#\#\#\#\#\#\#\#\#\#\#\#\#\#\#\#\#\#\#\#\#\#\#\#\#\#\#\#\#\#\#\#\#\#\#\#\#\#\#\#\#\#\#\#\#\#\#\#\#\#\#\#\#\#\#\#\#\#\#\#\#\#\#

Allez controler le résultat dans votre output hdfs et passez-le en local.






\question{Spark}\label{new-question}

Maintenant nous allons effectuer une requête sur la base de donné de hive avec spark. 


\lstinputlisting[language=java]{Scripts/spark.sh}
\#\#\#\#\#\#\#\#\#\#\#\#\#\#\#\#\#\#\#\#\#\#\#\#\#\#\#\#\#\#\#\#\#\#\#\#\#\#\#\#\#\#\#\#\#\#\#\#\#\#\#\#\#\#\#\#\#\#\#\#\#\#\#\#\#\#\#\#\#\#\#\#\#\#\#\#\#\#\#\#\#\#\#\#\#\#\#\#\#\#\#\#\#\#
\\Solution :
%%%%%%%%%%%%%%%%%%%%%%%%% SOLUTION 5%%%%%%%%%%%%%%%%%%%%%%%%%%%%%%%%
\lstinputlisting[language=java]{Scripts/sparksol.sh}
%%%%%%%%%%%%%%%%%%%%%%%%%%%%%%%%%%%%%%%%%%%%%%%%%%%%%%%%%%%%%%%%%%%
\#\#\#\#\#\#\#\#\#\#\#\#\#\#\#\#\#\#\#\#\#\#\#\#\#\#\#\#\#\#\#\#\#\#\#\#\#\#\#\#\#\#\#\#\#\#\#\#\#\#\#\#\#\#\#\#\#\#\#\#\#\#\#\#\#\#\#\#\#\#\#\#\#\#\#\#\#\#\#\#\#\#\#\#\#\#\#\#\#\#\#\#\#\#



Maintenant il s'agit de lancer ce code sur le cluster DAPLAB. : Downloader le projet github sur votre compte local de Daplab.


\lstinputlisting[language=sh,mathescape]{Scripts/git.sh}

Et finalement lancez le programme au moyen de spark-submit en passant par yarn.

\lstinputlisting[language=sh,mathescape]{Scripts/yarn.sh}


%sbt-package
%spark-submit --class "hivespark" --master yarn-client --num-executors 4 --executor-cores 4 --executor-memory 8g hivespark_2.11-1.0.jar


% \question{Prérequis}\label{new-question}
% This is not necessary, but it does give you a way of linking to a different question. In order to link to another question you simply need to add the following:
% 
% 
% 
% 
% 
% \begin{verbatim}
% \hyperref[question-label]{click here}
% \end{verbatim}
% 
% The first part \texttt{[question-label]} is the label name and the second part \texttt{\{click here\}} is the text that is displayed as link.
% 
% %------------------------------------------------
% 
% \question{How do I change the title and subtitle?}\label{change-title}
% 
% To change the main title, simply find the "TITLE AND LIST OF QUESTIONS" block and replace "A Template for FAQ's" within it. To change the subtitle find the following command:
% 
% \begin{verbatim}
% \newlistof{questions}{faq}{\large List of Frequently Asked Questions}
% \end{verbatim}
% 
% and replace the subtitle with one of your choosing.
% 
% %------------------------------------------------
% 
% \question{Is it possible to change the spacing between the questions in the list of questions?}\label{change-spacing}
% 
% Yes, simply find the following line:
% 
% \begin{verbatim}
% \setlength\cftparskip{.3em}
% \end{verbatim}
% 
% and change the \texttt{.3em} to whatever suits your fancy.
% 
% %------------------------------------------------
% 
% \question{What if I want to hide the page numbers and/or trailing dots next to the question in the list of questions?}\label{page-numbering}
% 
% To remove the trailing dots to the page numbers, find the line:
% 
% \begin{verbatim}
% %\renewcommand{\cftdot}{}
% \end{verbatim}
% and uncomment it. To remove the page numbers as well, find the following lines and uncomment them:
% \begin{verbatim}
% %\let\Contentsline\contentsline
% %\renewcommand\contentsline[3]{\Contentsline{#1}{#2}{}}
% \end{verbatim}
% 
% %------------------------------------------------
% 
% \question{Is it possible to number questions?}\label{number-questions}
% 
% Yes, you can refer to the number of the current question with:
% 
% \begin{verbatim}
% \thequestions
% \end{verbatim}
%  
% For example, this is question \thequestions. You can even incorporate question numbers into the questions and list of questions automatically by adding:
% 
% \begin{verbatim}
% Question \thequestions:
% \end{verbatim}
% 
% just before each \texttt{\#1} in the \texttt{\textbackslash questions} definition block in the preamble.
% 
% %------------------------------------------------
% 
% \question{Question \thequestions: Can I change the color of the question boxes?}\label{question-color}
% 
% Just find the following line and change the color specified there:
% 
% \begin{verbatim}
% \todo[inline, color=green!40]{\textbf{#1}}
% \end{verbatim}
% 
% %----------------------------------------------------------------------------------------

\end{document}